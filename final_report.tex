\documentclass{article}
\input{structure.tex}

\date{}

\begin{document}
	\renewcommand{\arraystretch}{1.5}
	
	\large
	\begin{titlepage}
		\begin{center}
			ФЕДЕРАЛЬНОЕ ГОСУДАРСТВЕННОЕ АВТОНОМНОЕ \\ ОБРАЗОВАТЕЛЬНОЕ УЧРЕЖДЕНИЕ \\ ВЫСШЕГО ПРОФЕССИОНАЛЬНОГО ОБРАЗОВАНИЯ \\ «НАЦИОНАЛЬНЫЙ ИССЛЕДОВАТЕЛЬСКИЙ УНИВЕРСИТЕТ \\ «ВЫСШАЯ ШКОЛА ЭКОНОМИКИ»
			
			\vspace{1cm}
			
			\textbf{Московский институт электроники и математики им. А.Н. Тихонова}
			
			\vspace{2cm}
			
			Рожин Андрей Константинович, группа БИТ 212
			
			\vspace{1.5cm}
			
			\textbf{Оценка аренды квартиры}
			
			\vspace{1cm}
			
			Курсовой проект \\
			по дисциплине «Алгоритмизация и программирование»
			
			\vspace{1cm}
			
			студента образовательной программы бакалавриата \\
			«Инфокоммуникационные технологии и системы связи»
		
			\vspace{1cm}
			\begin{flushright}
				Студент \underline{\hspace{5cm}}  \underline{А.К. Рожин}
			\end{flushright}
		
			\vspace{0.5cm}	
			
		\begin{flushright}
				\begin{tabular}{r} 
				 Научный руководитель \\ [-0.35cm]
				 к.т.н., доцент \\ [-0.35cm]
				 \underline{И. В. Назаров} \\
				 \\ \cline{1-1} 
				\end{tabular}
		\end{flushright}
			
			
			\vfill
			
			Москва
			2022
		\end{center}
	\end{titlepage}


	\newpage
	\thispagestyle{empty}

	\section* {Аннотация}
		\vspace{2ex}
		Разработана система полного цикла оценивания стоимости аренды квартиры в городе Москве. В частности - создан парсер данных, скрипт предобработки, написаны алгоритмы машинного обучения (без использования готовых решений от сторонних разработчиков), исследованы зависимости в данных, проведен EDA, обучена и полностью готова к использованию система машинного обучения.
		
		Благодаря возможностям API Telegram, создана оболочка взаимодействия конечного пользователя со всеми вышеперечисленными компонентами системы. 
		
		В открытом доступе существуют похожие решения, но они не отвечают высоким стандартам удобства использования, точности прогнозов и масштабируемости системы.
	\section* {Annotation}
		\vspace{2ex}
		A system of a full cycle assessment of the cost of renting an apartment in Moscow has been developed. In particular, a data parser was created, a preprocessing script, machine learning algorithms were written (without using ready-made solutions from third-party developers), data dependencies were investigated, EDA was conducted, a machine learning system was trained and fully ready for use.
		
		Thanks to the capabilities of the Telegram API, a shell of end-user interaction with all of the above system components has been created. 
		
		There are similar solutions in the public domain, but they do not meet the high standards of usability, accuracy of forecasts and scalability of the system.
		
	\newpage
	\thispagestyle{empty}
	\tableofcontents

	\newpage
	\setcounter{page}{4}
	\section {Обоснование выбора языка программирования и средств разработки}

		Проект разрабатывался на языке \textit{Python 3.8.8} \cite{litlink1}. Он позволяет встраивать в себя программы написанные на языке С/С++, упрощая интерфейс взаимодействия с ними. Также, это основный язык для написания оболочки взаимодействия с API Telegram и возможность взаимодействия с ядром \textit{Jupyter Notebook} \cite{litlink2}, что не менее важно в задачах анализа данных.
		
		Основой всего проекта стала библиотека \textit{NumPy} \cite{litlink3} --- проект с открытым исходным кодом, который упрощает работу с массивами и включает в себя сложные операции линейной алгебры. 
		
		Для хранения информации о квартирах был выбран \textit{Microsoft Excel} \cite{litlink4} --- простой и удобный интерфейс. В рамках моего проекта достаточно использования такой базы данных.
		
		Весь проект был разработан в интерактивной среде разработке \textit{IDE PyCharm 2021} \cite{litlink5}, которая обладает широким функционалом возможностей. Помимо этого, эта среда полностью настроена под язык \textit{Python}.
		
		Для контроля версий использовался \textit{Git} \cite{litlink6}.
		
		Также, при написании алгоритмов машинного обучения я пользовался книгой Яна Гудфеллоу --- Глубокое Обучение \cite{litlink7}, которая хоть и направлена на теорию по нейросетям, но для меня она стала источником новых знаний и концептуальных идей, которые я использовал в этом проекте.

	\newpage
	\section {Описание сценария}
	
		\begin{figure}[H]
			\centering
			\subcaptionbox{}{\includegraphics[width=0.9\textwidth]{img/new_img/pdf/scenery_description}}%
			\caption{Сценарий}
			\label{fig:scenery_description}
		\end{figure}
	
		На Рис. \ref{fig:scenery_description} представлен сценарий взаимодействия пользователя с программой. Ввод предполагаемой цены от пользователя --- опциональная возможность, которую можно пропустить.
		
	\newpage
	\section {Спецификация интерфейса}

		\begin{figure}[H]
			\centering
			\subcaptionbox{}{\includegraphics[width=0.25\textwidth]{img/old_img/eps-pdf/start_window}}%
			\hfill % <-- Seperation
			\subcaptionbox{}{\includegraphics[width=0.25\textwidth]{img/old_img/eps-pdf/comand_window}}%
			\caption{Элементы взаимодействия}
			\label{fig:start_element}
		\end{figure}
	
		На Рис.\ref{fig:start_element} (a) мы видим диалоговое окно программы, описание бота и кнопку \textit{Начать}, которая отправит программе сообщение \textit{start} и запустится сценарий, показанный на Рис.\ref{fig:scenery_description}. Запустить этот сценарий можно иначе --- на Рис.\ref{fig:start_element} (b) показано \textit{Меню команд}, нажав на которое можно в ручную отправить стартовое сообщение.
		
		\begin{figure}[H]
			\centering
			\subcaptionbox{}{\includegraphics[width=0.25\textwidth]{img/old_img/eps-pdf/inline_keyboard}}%
			\hfill % <-- Seperation
			\subcaptionbox{}{\includegraphics[width=0.25\textwidth]{img/old_img/eps-pdf/default_keyboard}}%
			\caption{Клавиатуры}
			\label{fig:keyboard}
		\end{figure}
		
		На Рис.\ref{fig:keyboard} показаны 2 типа клавиатур, которые будет использовать пользователь помимо встроенной в его систему. У них есть свои внутренние названия --- \textit{инлайн и дефолтная} клавиатуры. С точки зрения пользователя они отличаются только местонахождением на экране --- одна находится в самом диалоге, другая там же, где встроенная в систему клавиатура.
		

	\newpage
	\section {Структура программы}
	
		Для удобства, декомпозируем программу на 4 части, которые будут подсистемами одной большой системы --- оболочки взаимодействия.
		
		\begin{figure}[H]
			\centering
			\subcaptionbox{}{\includegraphics[width=0.8\textwidth]{img/new_img/pdf/all_modules}}%
			\caption{Основные блоки программы}
			\label{fig:all_modules}
		\end{figure}
	
		\subsection{Парсер}
		
			Парсер --- скрипт для сбора информации о квартире. Он написан на двух фреймворках --- \textit{Selenium} \cite{litlink8} и \textit{BeatifulSoup4} \cite{litlink9}. Первый отвечает за запуск браузера и сбор HTML разметки, второй за преобразование этой разметки в удобный для обработки вид.
			
			\begin{figure}[H]
				\centering
				\subcaptionbox{}{\includegraphics[width=0.8\textwidth]{img/new_img/pdf/parser}}%
				\caption{Структура парсера}
				\label{fig:parser}
			\end{figure}
		
		\newpage
		\subsection{Предобработка}
		
			Предобработка данных --- один из самых важных этапов построения пайплайна обучения. Все модели машинного обучения работают только с числами, они не способны определять номер дома, район и так далее. Как следствие, после сбора информации преобазовать ее в вид понятный компьютеру --- убрать названия домов, улиц и многое другое.	
			
			\begin{figure}[H]
				\centering
				\subcaptionbox{}{\includegraphics[width=0.5\textwidth]{img/new_img/pdf/preprocessing}}%
				\caption{Структура предобработки}
				\label{fig:preprocessing}
			\end{figure}
			
	
	
		\newpage
		\subsection{ML-модели}
			
			Как известно, в этом проекте не использовались реализации алгоритмов из библиотек, например, из \textit{SkLearn}. Решающие деревья и их ансамбли писались в ручную. Ниже представлена схема взаимодействия с реализацией алгоритмов машинного обучения.
			
			\begin{figure}[H]
				\centering
				\subcaptionbox{}{\includegraphics[width=0.8\textwidth]{img/new_img/pdf/ml_model}}%
				\caption{Структура моделей машинного обучения}
				\label{fig:ml_model}
			\end{figure}
		
			На Рис.\ref{fig:ml_model} представлена схема интерфейса взаимодействия с кодом, написанным на \textit{Cython} \cite{litlink10} --- библиотеки, которая компилирует код написанный на \textit{Python} в код на \textit{C} с использованием макросов. Уместить всю структуру блока \textit{ML-модели} довольно сложно, поэтому была выбрана такая реализация блок-схемы.
			
		\newpage
		\subsection{Взаимодействие с CIAN}
		
			Одной из особенностей программы является \textit{возможность оценить аренду в 2 клика}. Пользователю непотребуется самостоятельно вводить параметры квартиры, будет достаточно прислать ссылку на квартиру с сайта CIAN. Я вынес эту возможность в отдельный блок, так как не смотря на то, что работает она на тех же компонентах --- парсер, предобработка, модели машинного обучения, ее реализация потребовала значительного изменения этих блоков.
			
			\begin{figure}[H]
				\centering
				\subcaptionbox{}{\includegraphics[width=0.8\textwidth]{img/new_img/pdf/cian}}%
				\caption{Структура взаимодействия с CIAN}
				\label{fig:cian}
			\end{figure}
	
	\newpage
	\section{Описание алгоритма}	
		
		Алгоритм формирования БД для обучения.
		
		\begin{figure}[H]
			\centering
			\subcaptionbox{}{\includegraphics[width=0.8\textwidth]{img/new_img/pdf/algorithm}}%
			\caption{Парсинг}
			\label{fig:algorithm_parsing}
		\end{figure}
	
	\newpage
	\section{Заключение}
		Создал систему полного цикла, состоящую из парсера, модуля работы с данными и моделей машинного обучения, прописал сценарии использования системы при общении с телеграмм-ботом. На данном этапе развития, программа способна предсказывать цену в городе Москве в ценовом диапазоне до 100 тыс.руб / мес. Программа обеспечивает зонирование 90 \% районов Москвы для предсказания. В программу встроена функция просмотра зависимостей между различными величинами, например, как зависит логарифм времени, которое нужно затратить, чтобы добраться пешком до метро, от кол-ва комнат. Данная возможность встроена в исходный код, но в итоговой версии убрана, так как, по моему мнению, является лишней. Также при проектировании системы задумывалось создать FAQ по использованию программы --- эта возможность полностью реализована и встроена в исходный код, но убрана из итоговой версии, так как, по моему мнению, является лишней. В процессе разработки основной упор был сделан на упрощение интерфейса и его интуитивность, поэтому пришлось отказаться от некоторых идей, но их реализацию можно найти во внутренних файлах программы.
		
		\subsection{Перспективы}
			Несколько вариантов того, как улучшить текущую версию программы.
			
			\vspace{1cm}
			\textit{Функциональные:}
			
			\begin{enumerate}
				\item Спам-фильтр квартир - по параметрам и его изображению определить достоверность объявления.
				\item Создание приложения с рекомендательной системой, которая будет подсказывать квартиры по различным параметрам пользователя.
			\end{enumerate} 
		
			\textit{Технические:}
			\begin{enumerate}
				\item Подключить базу-данных SQL.
				\item Получить доступ к базе квартир, чтобы строить более четкие прогнозы.
				\item Написать некоторые элементы кода на С++, перебросить их на сервер и общаться через интернет соединение, чтобы не тормозила машина, на которой запущен бот.
			\end{enumerate} 
		
		
		
		
		
	\newpage
	%далее сам список используевой литературы
	\begin{thebibliography}{}
		\bibitem{litlink1}  Python release Python 3.8.8 // Python URL: \\ https://www.python.org/downloads/release/python-388/ 
		\bibitem{litlink2}  Jupyter Notebook Home // JN URL: \\ https://jupyter.org/
		\bibitem{litlink3}  NumPy Get Started // NP URL: \\ https://numpy.org/
		\bibitem{litlink4}  Microsoft Excel Official // ME URL: \\ https://www.microsoft.com/ru-ru/microsoft-365/excel
		\bibitem{litlink5}  PyCharm The Complete Package // PC URL: \\ https://www.jetbrains.com/pycharm/
		\bibitem{litlink6}	GIT --distributed-is-the-new-centralized // GIT URL: \\
		https://git-scm.com/
		\bibitem{litlink7}  \textbf{Я. Гудфеллоу, И. Бенджио, А. Курвилль} Глубокое обучение / пер. с анг. А.А. Слинкина. - 2-е изд., испр. - М.:ДМК Пресс, 2018. - 652 с.:цв.ил. 
		\bibitem{litlink8}  Selenium Chrome Python // SCP URL: \\
		https://selenium-python.readthedocs.io/
		\bibitem{litlink9}  BeatifulSoup4 Python // BS URL: \\
		https://pypi.org/project/beautifulsoup4/
		\bibitem{litlink10} Cython C-Extensions for Python // C URL: \\
		https://cython.org/
	\end{thebibliography}
\end{document}