\documentclass[a4paper,12pt]{article}

%%% Работа с русским языком
\usepackage{cmap}					% поиск в PDF
\usepackage{mathtext} 				% русские буквы в фомулах
\usepackage[T2A]{fontenc}			% кодировка
\usepackage[utf8]{inputenc}			% кодировка исходного текста
\usepackage[english,russian]{babel}	% локализация и переносы

\usepackage{hyperref} 

%-------Definition of \myrule--
\def\myrule#1#2#3{{\hskip#1in{\hbox to #2in%
			{\leaders\hbox to .00625in{\hfil.\hfil}\hfill}}%
		\par\hskip#1in#3\vskip1cm}}
%------------------------------


%%% Управление цветом
\usepackage[usenames]{color}
\usepackage{colortbl}

%%% Дополнительная работа с математикой
\usepackage{amsfonts,amssymb,amsthm,mathtools} % AMS
\usepackage{amsmath}
\usepackage{icomma} % "Умная" запятая: $0,2$ --- число, $0, 2$ --- перечисление

%% Номера формул
%\mathtoolsset{showonlyrefs=true} % Показывать номера только у тех формул, на которые есть \eqref{} в тексте.

%% Шрифты
\usepackage{euscript}	 % Шрифт Евклид
\usepackage{mathrsfs} % Красивый матшрифт

%% Свои команды
\DeclareMathOperator{\sgn}{\mathop{sgn}}

%% Перенос знаков в формулах (по Львовскому)
\newcommand*{\hm}[1]{#1\nobreak\discretionary{}
	{\hbox{$\mathsurround=0pt #1$}}{}}

%%% Работа с картинками
\usepackage{graphicx}  % Для вставки рисунков
\graphicspath{images}  % папки с картинками
\setlength\fboxsep{3pt} % Отступ рамки \fbox{} от рисунка
\setlength\fboxrule{1pt} % Толщина линий рамки \fbox{}
\usepackage{wrapfig} % Обтекание рисунков и таблиц текстом

%%% Работа с таблицами
\usepackage{array,tabularx,tabulary,booktabs} % Дополнительная работа с таблицами
\usepackage{longtable}  % Длинные таблицы
\usepackage{multirow} % Слияние строк в таблице

%%% Работа со стрелками
\usepackage{tikz-cd}


%%% Работа с исходным кодом 
\usepackage{minted}


\title{Техническое задание \\ на курсовой проект по дисциплине \\ 
«Алгоритмизация и программирование» \\ на тему «Оценка аренды квартиры»}

\date{}


\begin{document}
	
	\maketitle
	\thispagestyle{empty}
	
	\textbf{Шифр работы:} 30021002
	
	\vspace{5ex}
	\textbf{Цель работы:} создание модели машинного обучения, предсказывающую стоимость ежемесячной арендной платы за квартиру в Москве и интегрируемой с API Telegram.
	
	\vspace{5ex}
	\textbf{Требования к программе:}
	
	1. Команда, запускающая парсинг данных.
	
	2. Предобработка данных - заполнение пропущенных значений, добавление информации о районе (кафе, музеи, фитнес-клубы и т.д.), расстояние до центра города и т.д. 
	
	3. Определение координат квартиры  через API Geopy по адресу, расстояние до ближайшего метро и т.д.
	
	4. Получение от пользователя информации о квартире (самостоятельный ввод параметров, ссылка с ЦИАНа).
	
	5. Самостоятельная реализация нескольких (наиболее подходящих для этой задачи) алгоритмов обучения. Вывод результатов предсказания (результат может быть не один, а полученный из нескольких моделей, их композиций и других имплементаций).
	
	\newpage
	\thispagestyle{empty}
	
	6. Визуализация различных зависимостей признаков квартиры (например, как зависят кол-во комнат и район квартиры от арендной платы и наоборот)
	
	
	\vspace{2ex}
	\textit{Надежность:}
	
	1. Проверка входных данных на нужный тип и формат.
	
	2. Запись логов в .txt файл
	
	3. Автоматическая перезагрузка программы в случае ошибки
	
	\vspace{5ex}
	\textbf{Требования к составу технических средств:}
	
	\vspace{1ex}
	
	Процессор: Intel(R) Core(TM) i7-2630QM CPU @ 2.00GHz 
	
	Видеокарта: NVIDIA GeForce GTM 540M
	
	Объем встроенной памяти: 5ГБ
	
	Оперативная память: 6,00 ГБ
	
	Google Chrome - нужную версию смотреть в документации.
	
	Наличие аккаунта в Telegram.
	
	
	\vspace{5ex}
	\textbf{Ожидаемые результаты:}
	рабочая версия программы, документация к
	программе: техническое задание, итоговый отчет.
	
	\vspace{5ex}
	\textbf{Требования к информационной и программной совместимости:}
	
	\vspace{2ex}
	\textit{Операционная система}: Ubuntu 18.04+, Windows 10 21H2 и выше
	
	\vspace{2ex}
	\textit{Cреда программирования}: PyCharm Community Edition 2021.3.1.
	
	\vspace{2ex}
	\textit{Язык программирования}:
	
	В системе должен быть предустановлен Python 3.9 и Anaconda 3.\textit{Установка дополнительных библиотек, используемых в проекте, не потребуется, так как создано виртуальное окружение.}
	
	\newpage
	\thispagestyle{empty}
	\textit{Используемые библиотеки:}
	
	
	\textbf{Pandas} - работа с табличными данными.
	
	\textbf{NumPy} - работа с массивами, матрицами.
	
	\textbf{Aiogram} - обращение к API Telegram.
	
	\textbf{GeoPy} - геокодирование.
	
	\textbf{Cython} - ускорение вычислений.
	
	\textbf{SeaBorn} - визуализация данных
	
	\textbf{Altair} - визуализация данных
	
	
	\vspace{3ex}
	\textbf{Исполнители работы:} Рожин А.К.
	
	\vspace{5ex}
	\textit{Ответ на вопрос, почему модель выдала такой результат, даваться не будет.} 
	
	\textit{Автор программы не несет ответственности за результат, выданный моделью.}
	
	\textit{Получить право изменять исходный код любого модуля программы - невозможно.}
	
	
	
	
	\newpage
	\thispagestyle{empty}
	
	\textbf{Календарный план работы:}
	\vspace{2ex}
	
	\begin{tabular}{|>{\centering\arraybackslash}m{1.2cm}|>{\centering\arraybackslash}m{3cm}|>{\centering\arraybackslash}m{3cm}|>{\centering\arraybackslash}m{3cm}|>{\centering\arraybackslash}m{3cm}|}
		\hline
		№
		этапа & Название этапа & Входная
		информация
		этапа & Сроки
		исполнения & Форма
		отчетности \\
		\hline
		0 & Выдача задания &  & 10.01.2022 &  \\
		\hline
		1 & Формирование команды.
		Выбор темы. Разработка
		проекта технического задания
		(ТЗ) &  & По 24.01.2022 & Проект ТЗ \\
		\hline
		2 & Разработка сценария,
		структуры, алгоритма
		программы. Разработка
		спецификации интерфейса & Проект ТЗ & По 14.02.2022 & Проекты:
		сценария,
		структуры,
		алгоритма
		программы,
		спецификаци
		и интерфейса \\
		\hline
		3 & Корректировка ТЗ & Проект ТЗ & По 28.02.2022 & ТЗ на
		курсовой
		проект
		(распечатка в
		2 экз. с
		подписями) \\
		\hline 
		4 & Отладка программы & ТЗ & По 18.04.2022 & Альфа-
		версия
		программы,
		промежуточн
		ый отчет \\
		\hline
		5 & Защита курсового проекта & & 16.05.2022
		–6.06.2022 & Итоговый
		отчет (с ТЗ),
		программа \\
		\hline
		6 & Презентация & & 6.06.2022–
		13.06.2022 & Презентация
		разработки
		(публичное
		выступление) \\
		\hline
	\end{tabular}
	
	
	\newpage
	\thispagestyle{empty}
	
	\textbf{Исполнитель:}
	
	(БИТ212) Рожин А.К.
	
	\vspace{5ex}
	\myrule{0}{2.05}{подпись, дата}
	
	\vspace{-4.45cm}
	\begin{flushright}
		\textbf{Руководитель:}
		
		Назаров И.В.
	\end{flushright}
	
	\vspace{2.66ex}
	\myrule{3.15}{2.05}{подпись, дата}

	
	
	
	
	
\end{document}